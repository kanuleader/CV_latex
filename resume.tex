\documentclass{resume} 
\usepackage[left=0.75in,top=0.6in,right=0.75in,bottom=0.6in]{geometry} 
\newcommand{\tab}[1]{\hspace{.2667\textwidth}\rlap{#1}}
\newcommand{\itab}[1]{\hspace{0em}\rlap{#1}}
\name{Junam Song} % Your name
\address{kanuleader@gmail.com}

\begin{document}

\begin{rSection}{Education}


{\bf Kyungpook National University, South Korea} \hfill {\em March 2008 - August 2014} 
\\ B.S. in Electronics Engineering\hfill 

{\bf KAIST, South Korea} \hfill {\em September 2014 - August 2016} 
\\ M.S. in Electrical Engineering\hfill

\end{rSection}

\begin{rSection}{Career Objective}
 Building AI solutions to meet our needs.
\end{rSection}

\begin{rSection}{Projects}
{\bf Face detection in surveillance system }
\\ In video surveillance system, the exposure of a person’s face is a serious threat to personal privacy.
To protect the personal privacy in large amount of videos, an automatic face detection method is required to locate and mask the person’s face.

{\bf Low light image enhancement' technology on sensor ISP}
\\ Sensor image quality deteriorates due to noise in low-light environments. To solve this problem, an algorithm for improving image quality using multi-frames has been proposed.

{\bf Dual camera solution on sensor ISP}
\\ Image synthesis technology using heterogeneous dual cameras has been developed.

{\bf Deep learning based segmentation solution on mobile devices}
\\ Deep learning-based segmentation was developed and applied to maximize the bokeh effect in portrait mode.

{\bf Deep learning based image recovery solution to achieve yield improvements from a to z}
\\ As the pixels of the sensor become finer, the production yield decreases. Accordingly, a deep learning-based image restoration algorithm was developed to bring dead pixels to life.

\end{rSection}

\begin{rSection}{Technical Strengths}

\begin{tabular}{ @{} >{\bfseries}l @{\hspace{6ex}} l }
Languages \ & C, C++, Python, Matlab  \\
Technologies & Latex\\
Tools & Tensorlow, Tensorflow Lite, Pytorch, Caffe \\
Accelator & GPU, NPU, XNNPACK, NNAPI, Hexagon DSPs \\
Version Control & Github
\end{tabular}

\end{rSection}
% 
% 
% \newpage
\begin{rSection}{Work Experience}
\begin{rSubsection}{Samsung Electronics Intership Program,  South Korea}{January 2015 - January 2015}{Intern}{}
 \item During this period, I learned about sensor imaging processing techniques.
\end{rSubsection}
\begin{rSubsection}{Samsung Electronics, South Korea}{August 2016 - Present}{Engineer}{}
 \item The content below covers only stories related to deep learning and computer vision in my career.
 \item Face detection and tracking algorithms were used to design systems that automatically detect human faces in surveillance environments. In particular, deep learning-based face detector was ported to the C level and integrated into the system.
 \item As part of the application of the 2PD sensor, a Bokeh solution was developed by fusion of segmentation technology and the depth map using phase information. The architecture of segmentation was designed from previously proposed Deeplab. Quantization aware training additionally was used to highly optimize the architecture in mobile environment.
 \item On mobile, image recovery solution to defect data has been developed, and released to industrial field. I have been leading as a practical engineer to provide AI based solution including all need development process: Finding points where deep learning technology can be applied, defining the problem statements, creating a scenario for the overall system using SOTA technology in deep learning, and build optimized system using tensorflow lite and accelerator in mobile environment.
\end{rSubsection}

\end{rSection}
% 
%\newpage
\begin{rSection}{Achievements} 
 \item {\bf National Science and Engineering Undergraduate Scholorship} \hfill {\em March 2008 - February 2014} 
\end{rSection}

\begin{rSection}{Publications}
 \item Junam Song, Seung Ho Lee, Hyung-Il Kim, and Yong Man Ro "Fast Face Detection Robust to Low Illumination for Privacy Protection in Large-scale Surveillance Video," Korea Multimedia society, vol. 18, no. 2, pp. 30-33, Nov. 2015.
  \item Junam Song, Hyung-Il Kim, and Yong Man Ro "Robust and Fast Face Detection using CNN based Facial Component Heat Map and Face Bound Regression," Journal of Korea Multimedia society, vol. 19, no. 8, pp. 1310-1319, August. 2016.
\end{rSection}



\end{document}
