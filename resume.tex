\documentclass{resume} 
\usepackage[left=0.75in,top=0.6in,right=0.75in,bottom=0.6in]{geometry} 
\newcommand{\tab}[1]{\hspace{.2667\textwidth}\rlap{#1}}
\newcommand{\itab}[1]{\hspace{0em}\rlap{#1}}
\name{Junam Song} % Your name
\address{kanuleader@gmail.com}

\begin{document}

\begin{rSection}{Education}


{\bf Kyungpook National University, South Korea} \hfill {\em March 2008 - August 2014} 
\\ B.S. in Electronics Engineering\hfill 

{\bf KAIST, South Korea} \hfill {\em September 2014 - August 2016} 
\\ M.S. in Electrical Engineering\hfill

\end{rSection}

\begin{rSection}{Career Objective}
 Building AI solutions to meet our needs.
 And I believe that the plausible methodology itself is meaningless if it is not possible to implement what was thought.
\end{rSection}

\begin{rSection}{Projects}
{\bf Face detection in surveillance system }
\\ In video surveillance system, the exposure of a person’s face is a serious threat to personal privacy.
To protect the personal privacy in large amount of videos, an automatic face detection method is required to locate and mask the person’s face.

{\bf Low light image enhancement technology on sensor ISP}
\\ Noise in low-light environments is a major cause of sensor image quality degradation. In order to improve the resulting image quality deterioration, a method using multiple frames was applied.

{\bf Dual camera solution on sensor ISP}
\\ Image synthesis technology using heterogeneous dual cameras has been developed.

{\bf Deep learning based segmentation solution on mobile devices}
\\ Deep learning-based segmentation has been developed and applied to get more bokeh effect in portrait mode.

{\bf Deep learning based image recovery solution to achieve yield improvements from a to z}
\\ As the pixels of the sensor become finer, the production yield decreases. Accordingly, a deep learning-based image restoration algorithm was developed to bring dead pixels to life.

\end{rSection}

\begin{rSection}{Technical Strengths}

\begin{tabular}{ @{} >{\bfseries}l @{\hspace{6ex}} l }
Languages \ & C, C++, Python, Matlab  \\
Technologies & Latex\\
Tools & Tensorlow, Tensorflow Lite, Pytorch, Caffe \\
Accelator & GPU, NPU, XNNPACK, NNAPI, Hexagon DSPs \\
Version Control & Github
\end{tabular}

\end{rSection}
% 
% 
% \newpage
\begin{rSection}{Work Experience}
\begin{rSubsection}{Samsung Electronics Intership Program,  South Korea}{January 2015 - January 2015}{Intern}{}
 \item It was time to understand the overall flow of image processing methods that are essential for mobile image sensors.

\end{rSubsection}
\begin{rSubsection}{Samsung Electronics, South Korea}{August 2016 - Present}{Engineer}{}
 \item Below, the description focuses on working experiences related to deep learning and computer vision.
 \item Face detection and tracking algorithms were used to design systems that automatically detect human faces in surveillance environments. In particular, deep learning-based face detector was ported to the C level and integrated into the system.
 \item As part of the application of the 2PD sensor, a Bokeh solution was developed by fusion of segmentation technology and the depth map using phase information. The architecture of segmentation was designed from previously proposed Deeplab. This is the first case of applying the quantization aware training methodology to highly optimize the architecture in a mobile environment.
 \item On mobile, image recovery solution to defect data has been developed, and released to industrial field. The following process was involved in commercializing AI solutions for the first time as a practical engineer: find points where deep learning technology can be applied, define problem statements, create scenarios for the entire system using SOTA technology in deep learning, and build optimized systems using tensorflow lite and accelerators in a mobile environment
\end{rSubsection}

\end{rSection}
% 
%\newpage
\begin{rSection}{Achievements} 
 \item {\bf National Science and Engineering Undergraduate Scholorship} \hfill {\em March 2008 - February 2014} 
\end{rSection}

\begin{rSection}{Publications}
 \item Junam Song, Seung Ho Lee, Hyung-Il Kim, and Yong Man Ro "Fast Face Detection Robust to Low Illumination for Privacy Protection in Large-scale Surveillance Video," Korea Multimedia society, vol. 18, no. 2, pp. 30-33, Nov. 2015.
  \item Junam Song, Hyung-Il Kim, and Yong Man Ro "Robust and Fast Face Detection using CNN based Facial Component Heat Map and Face Bound Regression," Journal of Korea Multimedia society, vol. 19, no. 8, pp. 1310-1319, August. 2016.
\end{rSection}



\end{document}
