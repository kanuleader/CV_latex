\documentclass{resume} 
\usepackage[left=0.75in,top=0.6in,right=0.75in,bottom=0.6in]{geometry} 
\newcommand{\tab}[1]{\hspace{.2667\textwidth}\rlap{#1}}
\newcommand{\itab}[1]{\hspace{0em}\rlap{#1}}
\name{Junam Song} % Your name
\address{kanuleader@gmail.com}

\begin{document}

\begin{rSection}{Education}


{\bf Kyungpook National University, South Korea} \hfill {\em March 2008 - August 2014} 
\\ B.S. in Electronics Engineering\hfill 

{\bf KAIST, South Korea} \hfill {\em September 2014 - August 2016} 
\\ M.S. in Electrical Engineering\hfill

\end{rSection}

\begin{rSection}{Career Objective}
 Building AI solutions to meet our needs.
\end{rSection}

\begin{rSection}{Projects}
{\bf Face detection in surveillance system }
\\-

{\bf Low light image enhancement' technology on sensor ISP}
\\-

{\bf Dual camera solution on sensor ISP}
\\-

{\bf Deep learning based segmentation solution on mobile devices}
\\-

{\bf Deep learning based image recovery solution to achieve yield improvements from a to z}
\\As the pixels of the sensor become finer, the production yield decreases. Accordingly, a deep learning-based image restoration algorithm is developed to bring dead pixels to life.

\end{rSection}

\begin{rSection}{Technical Strengths}

\begin{tabular}{ @{} >{\bfseries}l @{\hspace{6ex}} l }
Languages \ & C, C++, Python, Matlab  \\
Technologies & Latex\\
Tools & Tensorlow, Tensorflow Lite, Pytorch, Caffe \\
Accelator & GPU, NPU, XNNPACK, NNAPI, Hexagon DSPs \\
Version Control & Github
\end{tabular}

\end{rSection}
% 
% 
% \newpage
\begin{rSection}{Work Experience}
\begin{rSubsection}{Samsung Electronics Intership Program, Korea}{January 2015 - January 15}{Engineer}{}
 \item -
\end{rSubsection}
\begin{rSubsection}{Samsung Electronics, Korea}{August 2016 - Present}{Engineer}{}
 \item On mobile, image recovery solution to defect data has been developed, and released to industrial field. I have been leading as a practical engineer to provide AI based solution including all need development process: Finding points where deep learning technology can be applied, defining the problem statements, creating a scenario for the overall system using SOTA technology in deep learning, and build optimized system using tensorflow lite and accelerator in mobile environment.
\end{rSubsection}

\end{rSection}
% 
\newpage
\begin{rSection}{Achievements} 
 \item {\bf National Science and Engineering Undergraduate Scholorship} \hfill {\em March 2008 - February 2014} 
\end{rSection}

\begin{rSection}{Publications}
 \item Junam Song, Seung Ho Lee, Hyung-Il Kim, and Yong Man Ro "Fast Face Detection Robust to Low Illumination for Privacy Protection in Large-scale Surveillance Video," Korea Multimedia society, vol. 18, no. 2, pp. 30-33, Nov. 2015.
  \item Junam Song, Hyung-Il Kim, and Yong Man Ro "Robust and Fast Face Detection using CNN based Facial Component Heat Map and Face Bound Regression," Journal of Korea Multimedia society, vol. 19, no. 8, pp. 1310-1319, August. 2016.
\end{rSection}



\end{document}
